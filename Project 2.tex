\documentclass[10pt]{extarticle}
\usepackage[utf8]{inputenc}
\usepackage[fleqn]{amsmath}
\usepackage{lipsum}
\usepackage{xparse}
\usepackage{graphicx,wrapfig}
\usepackage[bottom=1in,top=1in,left=.75in,right=.75in]{geometry}
\usepackage{xcolor}
\usepackage{enumitem}
\usepackage{tabularx}
\usepackage{array}

\DeclareDocumentCommand{\start}{ O{Kevin Shao \& Jeffrey Cheng} m m m }{#1\\#2\\#3\\#4}
\DeclareDocumentCommand{\blank}{}{\textrm{ }}

\title{Multiply Methods of Classification in R:\\ Classifying Handwritten Numbers 0-9\\\textrm{ }\\\large Regression and Machine Learning F}
\date{16 April 2018}
\author{Kevin Shao \& Jeffrey Cheng}

\begin{document}
\allowdisplaybreaks
\hbadness=100001
\renewcommand{\labelitemi}{$>$}
\maketitle
\begin{center}

\includegraphics[scale=0.5]{numbers.png}

%\begin{tabular}{c c c}
%	\includegraphics{one.png} & \includegraphics{two.png} & \includegraphics{three.png}\\
%	\includegraphics{four.png} & \includegraphics{five.png} & \includegraphics{six.png}\\
%	\includegraphics{seven.png} & \includegraphics{eight.png} & \includegraphics{nine.png}
%\end{tabular}
\end{center}
\newpage
\noindent \start{Mr. Piper}{Machine Learning F}{14 May 2018}

\section{Abstract}

Our data set is the mnist data set, which is a series of 28x28 pixel images that represent the ten numbers, zero through nine. Thus, taking each pixel to be an explanatory variable, there are 784 explanatory variables for the response variable, the number that the pixels make up. Each explanatory variable is an integer from 0 to 255, which represents the gray-scaling of pixels. 0 is a white pixel, and 255 is a completely black pixel. Because the explanatory variables represent individual pixels within different images, it is not really helpful to perform a five number summary. 

\section{Preprocessing}

There was not much general preprocessing to be done, most of the data processing was specific to each classification method. The data was in a .csv format since images cannot be directly inputted and be classified. Our training data set consisted of 59999 observations while our test set had 9999 observations. All observations had 785 columns, the first column begin the true number followed by the rest of the pixel data. An example of each number is shown below:

\begin{center}
	\begin{tabular}{c c c c c c c c c c}
		\includegraphics{zero.png} & \includegraphics{one.png} & \includegraphics{two.png} 
		\includegraphics{three.png} & \includegraphics{four.png} & \includegraphics{five.png} 
		\includegraphics{six.png} &\includegraphics{seven.png} &\includegraphics{eight.png} 
		\includegraphics{nine.png}
	\end{tabular} 
\end{center}

\section{LDA}

There were two main problems when attempting to do lda on the dataset. Firstly, since the training and test sets are so big, the time it would take R to complete the lda classification would be substantially long. Thus to counteract this problem, we took a small subset of our training and test sets. Since they were already randomized, there is no need to take a random sample, thus we took the first 2000 observations of the training set and the first 500 observations of the test set.\\\\
The second problem is that some of the pixels remain white through all the images. So, when R attempts to do lda, it indicates that the variables are constant. To counteract this problem, we took the sums of the columns of the dataset, which corresponds to the total grayscale of a specific pixel throughout all 2000 training images. If the sum was greater than an arbitrary number, say 1000, we used the column in our lda. If the sum were less than that arbitrary number, then we removed it from the data set. Using 1000 as our arbitrary number for now, we obtained a test and training error and corresponding confusion matrices.
\begin{center}
	\begin{tabular}{c c}
		Training Error & Test Error\\
		0.041 & 0.246
	\end{tabular}\\
\end{center}
\begin{minipage}{.6\textwidth}
	\begin{center}
	\begin{tabular}{c | c c c c c c c c c c}
		d \textbackslash r&0&1&2&3&4&5&6&7&8&9\\ \hline
		0&186&0&1&0&0&0&0&0&1&0\\
		1&0&214&3&0&1&1&1&6&3&0\\
		2&0&0&187&0&0&1&0&0&0&0\\
		3&1&1&0&185&0&2&0&0&0&0\\
		4&0&0&2&0&203&1&0&1&0&6\\
		5&0&2&0&2&2&171&0&0&1&1\\
		6&1&0&0&0&1&1&199&0&1&0\\
		7&0&0&1&1&0&0&0&210&0&3\\
		8&3&2&3&0&1&2&0&0&165&2\\
		9&0&1&1&3&6&1&0&7&1&198
	\end{tabular}
	\bigskip 
	Training Set Confusion Matrix
	\end{center}
\end{minipage}
\begin{minipage}{.4\textwidth}
The d and the r refer to data and reference. So, the numbers running along the top of the matrix are the true numbers while the numbers along the left are what lda predicts. There are no consistent errors in the data set. The greatest errors come from confusing 4s with 9s, a task not teachers can consistently get right (for all those times that Dr. Brown couldn't read my handwriting).
\end{minipage}\\
\begin{minipage}{.45\textwidth}
	It is interesting how much lower the test error is than the training error. We thought this was due to the fact that the lda model that we created was still more tailored to the training set we created. If our training set were bigger, the discrepancy between training and test error would decrease. Nonetheless, the both errors we got, were much lower than the 90\% error we would obtain from randomly classifying.
\end{minipage}
\begin{minipage}{.05\textwidth}
	\quad
\end{minipage}
\begin{minipage}{.5\textwidth}
	\begin{center}
	\begin{tabular}{c | c c c c c c c c c c}
		d\textbackslash r & 0 & 1 & 2 & 3 & 4 & 5 & 6 & 7 & 8 & 9\\ \hline
		0&35&0&0&0&1&1&2&0&0&0\\
		1&0&66&6&0&1&1&1&1&1&1\\
		2&1&0&36&2&1&3&4&2&2&0\\
		3&1&0&1&34&1&3&0&1&1&1\\
		4&0&0&2&0&41&1&2&0&4&7\\
		5&3&1&1&5&0&36&2&0&2&2\\
		6&0&0&2&0&1&1&32&0&0&0\\
		7&0&0&2&1&0&0&0&33&2&5\\
		8&1&0&4&1&0&3&0&0&27&1\\
		9&1&0&1&3&9&1&0&11&1&37\\
	\end{tabular}
	\bigskip
	Test Set Confusion Matrix
	\end{center}
\end{minipage}	
We wondered if we could improve upon our error if we changed the arbitrary value of 1000 we initially chose. So, we wrote a function that takes in the value of for which the sum of pixels in the training set needs to be greater than and returns the test and training errors.\\\\
\begin{minipage}{.25\textwidth}
	\begin{tabular}{c c c}
		n & Training & Test\\ \hline
		50&0.0340&0.274\\
		100&0.0340&0.272\\
		150&0.0335&0.264\\
		200&0.0375&0.256\\
		250&0.0345&0.260\\
		300&0.0345&0.258\\
		350&0.0350&0.253\\
		400&0.0350&0.246\\
		450&0.0355&0.248\\
		500&0.0370&0.244
	\end{tabular}
\end{minipage}
\begin{minipage}{.25\textwidth}
	\begin{tabular}{c c c}
		n & Training & Test\\ \hline
		550&0.0365&0.240\\
		600&0.0390&0.243\\
		650&0.0385&0.245\\
		700&0.0385&0.238\\
		750&0.0395&0.238\\
		800&0.0400&0.242\\
		850&0.0405&0.244\\
		900&0.0410&0.244\\
		950&0.0410&0.244\\
		\textbf{1000}&\textbf{0.0410}&\textbf{0.246}\\
	\end{tabular}
\end{minipage}
\begin{minipage}{.25\textwidth}
	\begin{tabular}{c c c}
		n & Training & Test\\ \hline
		1050&0.0400&0.240\\
		1100&0.0400&0.244\\
		1150&0.0405&0.246\\
		1200&0.0405&0.242\\
		1250&0.0405&0.232\\
		1300&0.0420&0.236\\
		1350&0.0420&0.232\\
		1400&0.0425&0.236\\
		1450&0.0425&0.236\\
		1500&0.0425&0.238	
	\end{tabular}
\end{minipage}
\begin{minipage}{.25\textwidth}
	\begin{tabular}{c c c}
		n & Training & Test\\ \hline
		1550&0.0425&0.238\\
		1600&0.0435&0.238\\
		1650&0.0435&0.234\\
		1700&0.0450&0.234\\
		1750&0.0445&0.236\\
		1800&0.0445&0.236\\
		1850&0.0445&0.236\\
		1900&0.0450&0.232\\
		1950&0.0450&0.232\\
		2000&0.0455&0.230
	\end{tabular}
\end{minipage}\\\\
The bolded text represents the n = 1000 that we initially chose. There is a clear trend, the lower the n, which in turn represents less columns being discarded thus leading to a greater number of explanatory variables, the lower the training error. However, while the test error has a less linear trend, the general trend is the exact opposite of the training error. This makes sense since we are only using the most important explanatory variables when n is larger.\\\\
Is there another way to use lda to classify the variables? Since even using the reduced number of variables takes a long time for R to process, we wondered if the total number of colored pixels was related to the true number. After all, some numbers can be written using less pixels (read: ink) than others. Thus, we took a slightly larger training and test set, 10000 and 2500, respectively, and ran lda on the data.\\\\
\begin{minipage}{.45\textwidth}
\begin{center}
\begin{minipage}{.45\textwidth}
	\begin{tabular}{c c}
	number & fill\\ \hline
	0& 194.87712\\
	1&  89.11003\\
	2& 168.83653\\
	3& 165.40174\\
	4& 143.07959
	\end{tabular}
\end{minipage}
\begin{minipage}{.45\textwidth}
	\begin{tabular}{c c}
	number & fill\\ \hline
	5& 149.38631\\
	6& 157.02761\\
	7& 133.80748\\
	8& 173.62288\\
	9& 145.44683
	\end{tabular}
\end{minipage}
\begin{tabular}{c}
	Test Error\\
	0.778
\end{tabular}
\end{center}
\end{minipage}
\begin{minipage}{.6\textwidth}
	\begin{center}
	\begin{tabular}{c | c c c c c c c c c c}
	d\textbackslash r&0&1&2&3&4&5&6&7&8&9\\ \hline
	0&79&1&63&47&25&34&46&10&83&16\\
	1&2&274&17&22&52&15&15&98&6&31\\
	2&8&0&5&2&2&6&1&0&4&2\\
	3&69&0&53&52&32&40&32&17&62&31\\
	4&0&0&0&0&0&0&0&0&0&0\\
	5&0&0&0&0&0&0&0&0&0&0\\
	6&27&2&49&25&31&29&32&19&24&24\\
	7&24&10&85&96&124&88&90&109&59&134\\
	8&6&0&1&1&2&4&1&0&0&2\\
	9&4&0&4&9&7&5&8&3&4&4
	\end{tabular}\\
	\bigskip
	Test Set Confusion Matrix
	\end{center}
\end{minipage}\\\\
The test error is so high because the group means shown above are so close to each other. This is also reflected in the confusion matrix, not one observation in the test set was classified as either a 4 or 5. While some numbers, such as one, were consistently correctly classified due to its unique pixel count, there were too many numbers that had a pixel count of around 140, which led to severe misclassification.
\section{k-nn}
lol kevin
\section{Random Forest}
\end{document}